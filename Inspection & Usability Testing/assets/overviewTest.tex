In this part of the document we evaluate usability of Moviri basing on \textbf{User Testing}. User testing is a technique based on user-experience used to evaluate a product by testing it on selected users. Instead of inspection, this is more focused on human–computer interaction in order to eveluate the easyness of a website.

\section{Design of the study}
\subsection{User Profile}
User profile definition is very relevant for user testing. In fact, this help us to understand who can be recruited for user testing. The user profile detected are the following:

\medskip
\textbf{Profile 1}\par
\begin{itemize}
\item \textit{Age}: between 25 and 30 years old;
\item \textit{Civil state}: irrelevant;
\item \textit{Technology capabilities}: General knowledge about Computer Science;
\end{itemize}

\medskip
\textbf{Profile 2}\par
\begin{itemize}
\item \textit{Age}: between 40 and 50 years old;
\item \textit{Civil state}: irrelevant;
\item \textit{Technology capabilities}: Simple web user with any particular capabilities;
\end{itemize}

\subsection{Scenarios}
In this subsection are shown the goals of the user testing followed by the tasks needed to reach them:\par
\textbf{Scenario 1}: You need a job and you know that Moviri has a lots of open positions. You’re applying for a job opportunity.
\begin{enumerate}
\item Visit the website section for career service;
\item Start the application procedure;
\item Choose a type of job offer that is suitable for your background (such as Software Engineering); 
\item Select the position for which you want to run for;
\item Insert your data in the form for the application;
\item Turn to the homepage;
\end{enumerate}\par
\textbf{Scenario 2}: you’re looking for a white paper related to a particular service/solution for your company.
\begin{enumerate}
\item Visit the website section for consulting Moviri’s resource;
\item Find and select the white paper that you need;
\item Fill out the form with your information in order to receive the white paper by e-mail;
\item Turn to the homepage;
\end{enumerate}\par
\textbf{Scenario 3}: you’re looking for a initiative called “Keep IT Up” and you want to keep current on it.
\begin{enumerate}
\item Visit the website section relevant to news;
\item Find the article about “Keep IT Up” initiative and select it;
\item After you get some info from the article, in order to stay up fill out the form;
\end{enumerate}\par

\subsection{Variable Measure}
In order to evaluate user testing on some metrics, usability variables are defined:
\begin{itemize}
\item \textbf{Time of execution}: it's measure in \textit{seconds} and represents the time spent on a given task. It starts from the moment in which user begin focusing his attention on it;
\item \textbf{Effectiveness}: it represents the task success rate. It's measured by a value \textbf{between 0 and 1}:
    \begin{itemize} 
    \item \textit{1.0}: task completed with success;
    \item \textit{0.5}: task partially completed;
    \item \textit{0.0}: task is not completed;
    \end{itemize}
\item \textbf{Errors}: it's measured by an integer representing the number of errors made during the execution on a given task;
\item \textbf{Perceived tasks difficulty}: it's expressed by an integer \textit{between 0 and 5} given by an user after the completion of a given task;
\item \textbf{Satisfaction}: it's given by an integer \textit{between 0 and 5} provided by user immediately after the task execution;
\end{itemize}


\subsection{Final Survey}
After the execution of every tasks, a final survey is provided to users in order to have additional collected data. This provides extra feedbacks about aspects already highlighted during the inspection.
The questions, provided through a survey made with Google Forms at the link \url{www.ciao.123}, are the following:
\begin{enumerate}
\item Did you find virtual design intuitive and consistent?
\item Did you find images dimension good and well positioned?
\item Did you find the navigation between each sections appropriate?
\item How much easy was to navigate between different website domains due to task completion?
\item How much easy was to find landmarks?
\item How much easy was to carry out each task (in general)?
\end{enumerate}