In this part of the document we're focusing on the evaluation of usability of \textit{Moviri} through \textbf{inspection method}. Inspection allows us to find usability issues and obstacles for the user when interacting with a web application. In particular, this is done thanks to \textbf{heuristics} which guides the expert to explore the website and check compliance with usability principles.
\section{Goals}
Before the inspection, goals are defined in order to deeply inspect the website and to focus on the main aspect. 
\begin{itemize}
\item Read experiences of other companies and changes adopted;
\item Find the appropriate technology needed;
\item Interact with Moviri due to become a new customer;
\end{itemize}
\section{Inspection method}

We decide to adopt 2 different set of heuristics in order to inspect the website. 

\subsection{Mile's heuristics}
These heuristics are divided into different categories relevant to a particular aspect.

\textbf{Navigation}: It aims to evaluate the easiness with which an user navigates into each part of the website.
\begin{itemize} 
\item \textbf{Interaction consistency}: do pages of the same type have the same links and interaction capability?
\item \textbf{Group navigation}: is it easy to navigate from and among groups of
“items”?
\item \textbf{Structural Navigation}: is it easy to navigate among the semantic components of a Topic?
\item \textbf{Semantic Navigation}: is it easy to navigate among group members and from a group introductory page to group members (and the other way around)?
\item \textbf{Landmarks}: is it easy to navigate from a Topic to a related one?
\end{itemize}

\textbf{Contents}: It indicates how in the website information is well balanced in each page and section.  
\begin{itemize}
\item \textbf{Information Overload}: is the information in a page too much or too little and does it fit the page layout?
\end{itemize}

\textbf{Layout}: It serves to estimate if the website is graphically expressive enough and readable.
\begin{itemize}
\item \textbf{Text Layout}: is the text readable? Is font size appropriate?
\item \textbf{Interaction Placeholder-Semiotics}: are textual or visual labels of interactive elements “expressive”? i.e., do they reflect the meaning of the interaction and its effects? Are they consistent?
\item \textbf{Interaction Placeholders-Consistency}: are textual or visual labels of interactive elements consistent in terms of wording, icon, position, etc.?
\item \textbf{Spatial Allocation}: is the on-screen allocation of contents and visual appropriate for their relevance? Are “semantically related” elements close and “semantically distant” element far away?
\item \textbf{Consistency of Page Structure}: do pages of the same type have the same lay out (same visual properties of each component and similar lay-out organization of the various elements?)
\end{itemize} 

\subsection{Nielsen's heuristics}
These belongs to a set of 10 heuristics, which cover each aspect relevant for the evaluation. However, we decide not to evaluate every heuristic but only some of them; we decide this because in our opinion there are some of them that cannot applied to the website.
The heuristics are the following:
\begin{description} 
\item[H1 -] \textbf{Visibility of system status}: the design should always keep users informed about what is going on, through appropriate feedback within a reasonable amount of time;
\item[H2 -] \textbf{Match between system and the real world}: the design should speak the users' language. Use words, phrases, and concepts familiar to the user, rather than internal jargon. Follow real-world conventions, making information appear in a natural and logical order;
\item[H4 -] \textbf{Consistency and standards}: users should not have to wonder whether different words, situations, or actions mean the same thing. Follow platform and industry conventions
\item[H5 -] \textbf{Error prevention}: good error messages are important, but the best designs carefully prevent problems from occurring in the first place. Either eliminate error-prone conditions, or check for them and present users with a confirmation option before they commit to the action;
\item[H7 -] \textbf{Flexibility and efficiency of use}: shortcuts — hidden from novice users — may speed up the interaction for the expert user such that the design can cater to both inexperienced and experienced users. Allow users to tailor frequent actions;
\item[H8 -] \textbf{Aesthetic and minimalist design}: interfaces should not contain information which is irrelevant or rarely needed. Every extra unit of information in an interface competes with the relevant units of information and diminishes their relative visibility;
\end{description}


\section{Scoring metric}
Before the inspection, a metric is defined in order to evaluate each heuristic. The evaluation consist in the assignment of a score from 0 to 5. 

\begin{itemize}
\item \textbf{0}: The heuristics is not satisfied. Many severe violations are detected;;
\item \textbf{1}: The heuristics is not satisfied. Few severe violations are detected;
\item \textbf{2}: The heuristics is not satisfied because of relevant issues;
\item \textbf{3}: The heuristic is partially satisfied with some issues detected;
\item \textbf{4}: The heuristic is satisfied, but it can be improved;
\item \textbf{5}: No issue are detected. The heuristic is fully satisfied; 
\end{itemize}

