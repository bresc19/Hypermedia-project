\section{Navigation}
\begin{table}[H]
  \begin{center}
    \label{tab:table1}
    \begin{tabular}{l|c|r} % <-- Alignments: 1st column left, 2nd middle and 3rd right, with vertical lines in between
      \textbf{Heuristic} & \textbf{Score} & \textbf{Comment}\\
      
      \hline
      Interaction Consistency & 1110.1 & a\\
      Group Navigation & 10.1 & b\\
      Structural Navigation & 23.113231 & c\\
      Semantic Navigation & 23.113231 & c\\
      Landmarks & 23.113231 & c\\

    \end{tabular}
  \end{center}
\end{table}

\subsection{Interaction Consistency}
The website offers a significant experience due to interaction. As a matter of fact, its structure is based on some elements available in each page:
\begin{itemize}
\item \textbf{Header}: it links to different pages. In particular it allow users to navigate on:
\begin{itemize}
\item Company's \textbf{social network profiles} (\textit{Facebook, Twitter, Instagram and Linkedin});
\item \textbf{News section} for incoming report regarding Moviri;
\item \textbf{Contact section} due to interact with Moviri's employees;
\item \textbf{Moviri Careers}. It's a secondary website of Moviri where there are informations regarding job oppurtunities (+ genaral description of lavorare con moviri);
\end{itemize}
\item \textbf{Topbar}: allows a user to surf in each section of the website. In addition provides a search function for any content; 
\item \textbf{Footer}: provides the same links to different sections of topbar and header, but at the foot of each page;
\end{itemize}
\subsection{Group Navigation}
Thanks to components such as topbar and footer, is possibile to navigate between each section through links. The fact that links are placed both high and low helps user's experience to be more intuitive.
\subsection{Structural Navigation}
Navigation along each page is easily feasible and understandable. Structural navigation change slightly between section.\\
For instance, in the \textit{Business Lines} section each topic is represented by the combination of image and description side by side and unpaired with respect to the next one.
While in the \textit{Resource} section each item is placed in a grid with its title and a respective image.  

\subsection{Semantic Navigation}
Navigation between pages of different sections is easily allowed by topbar and footer. Nevertheless, there are situations in which this is not reversible. In particular this is not possibile in the \textit{Resource} section after the click of an item. In fact, it redirects to pages of other websites leaving Moviri domain. It's sufficient to go back with the undo function of the browser, but could be quite uncomfortable.
\subsection{Landmarks}
These can be found both top and buttom part of the website in each section. It's always possibile to be redirect to the homepage. Anyway they could be improved to be more evident.

\section{Content}
\begin{table}[H]
  \begin{center}
    \label{tab:table1}
    \begin{tabular}{l|c|r} % <-- Alignments: 1st column left, 2nd middle and 3rd right, with vertical lines in between
      \textbf{Heuristic} & \textbf{Score} & \textbf{Comment}\\
      
      \hline
     Information Overload & 1110.1 & a\\
     
    \end{tabular}
  \end{center}
\end{table}
\subsection{Information Overload}
Each information, both graphical and textual, is well balanced in each section of the website. This is one of the strength of website.
\section{Layout}

\begin{table}[H]
  \begin{center}
    \label{tab:table1}
    \begin{tabular}{l|c|r} % <-- Alignments: 1st column left, 2nd middle and 3rd right, with vertical lines in between
      \textbf{Heuristic} & \textbf{Score} & \textbf{Comment}\\
      
      \hline
      Text Layout & 1110.1 & a\\
      Interaction Placeholder-Semiotics & 10.1 & b\\
      Interaction Placeholder-Consistency & 23.113231 & c\\
      Spatial Allocation & 23.113231 & c\\
      Consistency of Page Structure & 23.113231 & c\\

    \end{tabular}
  \end{center}
\end{table}

\subsection{Text Layout}
Textual contents are easy-to-read. This is thanks to the font used in each point which is always proportional to the importance of the information. For instance, title or quotes has a larger font then simple description. An other important aspect is the choice of the font colour. In fact this is always matched with the rest of the layout.
\subsection{Interaction Placeholder-Semiotics}
Textual and visual labels are expressive almost in every case. Only few situations link are not visible by underling or a hand-cursor. In addition we found relevant issues caracterized by a strange behaviour in some case. For instance, clicking on partners image in the \textit{Business Line} section, the page come back at the top of the page. Moreover, it happens that cliking on image TODO.
\subsection{Interaction Placeholder-Consistency}
Website is consistent due to its component such as wording, icon and position. In fact, due to the easiness of the website, this has no relevant issue.

\subsection{Spatial Allocation}
Each type of content is allocated spatially and semantically very well. 

\subsection{Consistency of Page Structure}
The layout in each page is generally the same, with few differences about contents allocation. Anyway, in the resource section users can be redirected to other websites with a completely different structure.\\
TODO