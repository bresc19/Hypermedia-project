\section{Navigation}
\label{Navigation}
\begin{table}[H]
  \begin{center}
    \label{tab:table1}
    \begin{tabular}{||l|c|p{8cm}||} % <-- Alignments: 1st column left, 2nd middle and 3rd right, with vertical lines in between
      \textbf{Heuristic} & \textbf{Score} & \textbf{Comment}\\
      
      \hline
      Interaction Consistency & 4.5 & The website has a good interaction structure reflected in the pages of the same type. Elements as header, topbar and footer are present in every screen. There could be more consistency in links shape.\\
      \hline
      Group Navigation & 4.5 & Navigation from a list of items to its members is acceptably easy and intuitive, as well as between group members and from group member to the list.\\
      \hline
      Structural Navigation & 4 & It's acceptably intuitive between the components of a topic, well enough described and presented; it would have been more appropriate if it were equable between the sections.\\
      \hline
      Semantic Navigation & 3.5 & Even if there are almost always links to related contents, navigation between related topics is not always accessible in both directions.\\
      Landmarks & 3.5 & Present in the various pages of the website but which do not always identify clear and easy-to-read references\\

    \end{tabular}
  \end{center}
\end{table}

\subsection{Interaction Consistency}
The website offers a significant experience due to interaction structure reflected in the pages of the same type. As a matter of fact, its structure is based on some elements available in each page: 
\begin{itemize}
\item \textbf{Header} (figure \ref{Header}): it links to different pages. In particular it allow users to navigate on:
\begin{itemize}
\item Company's \textbf{social network profiles} (\textit{Facebook, Twitter, Instagram and Linkedin});
\item \textbf{News section} for incoming report regarding Moviri;
\item \textbf{Contact section} due to interact with Moviri's employees;
\item \textbf{Moviri Careers}. It's a secondary website of Moviri where there are informations regarding job oppurtunities (+ genaral description of lavorare con moviri);
\end{itemize}
\item \textbf{Topbar} (figure \ref{topbar}): allows a user to surf in each section of the website. In addition provides a search function for any content; 
\item \textbf{Footer} (figure \ref{Footer}): provides the same links to different sections of topbar and header, but at the foot of each page;
\end{itemize}
At the end, there could be more consistency in links shape.	


\begin{figure}[H]
  \centering
  \makebox[\linewidth]{
    \includegraphics[scale=0.35]{images/header.png}}
  \caption{Header.}
   \label{Header}
\end{figure}

\begin{figure}[H]
  \centering
  \makebox[\linewidth]{
    \includegraphics[scale=0.35]{images/topbar.png}}
  \caption{Topbar.}
   \label{topbar}
\end{figure}

\begin{figure}[H]
  \centering
  \makebox[\linewidth]{
    \includegraphics[scale=0.35]{images/footer.png}}
  \caption{Footer.}
   \label{Footer}
\end{figure}

\subsection{Group Navigation}
The navigation between the various components of a topic is intuitive, thanks to the good arrangement of the parts, that almost always have a title, an image and a short description. However, structural navigation slightly change between the various sections. For instance, in the \textit{Business Lines} section each topic is represented by the combination of image and description side by side and unpaired with respect to the next one, while in the \textit{Resource} section each item is placed in a grid with its title and a respective image.
\subsection{Structural Navigation}
Thanks to components such as topbar and footer, is possibile to navigate between each section through links. The fact that links are placed both high and low helps user’s experience to be more intuitive, since it is not possible to go to the previous or next page. Anyway the Structural Navigation along each page is always easily feasible and understandable. 

\begin{comment}
Navigation along each page is easily feasible and understandable. Structural navigation change slightly between section.\\
For instance, in the \textit{Business Lines} section each topic is represented by the combination of image and description side by side and unpaired with respect to the next one.
While in the \textit{Resource} section each item is placed in a grid with its title and a respective image.  
\end{comment}


\subsection{Semantic Navigation}
Navigation between pages of different sections is easily allowed by topbar and footer. Nevertheless, there are situations in which this is not reversible. In particular this is not possibile in the \textit{Resource} section after the click of an item; in fact, it redirects to pages of other websites leaving Moviri domain. It's sufficient to go back with the undo function of the browser, but could be quite uncomfortable.
\subsection{Landmarks}
Landmarks cover all major contents of the website and they can be found both top and buttom part of the website in each section; Anyway, they could be improved to be more clear, evident and easily readable. It’s always possibile to be redirect to the homepage but the reference is not immediately understandable for a user visiting the website for the first time, considering that it is not identified with the well recognized term \textbf{“Home”}.


\section{Content}
\begin{table}[H]
  \begin{center}
    \label{tab:table1}
    \begin{tabular}{||l|c|p{8cm}||} % <-- Alignments: 1st column left, 2nd middle and 3rd right, with vertical lines in between
      \textbf{Heuristic} & \textbf{Score} & \textbf{Comment}\\
      
      \hline
     Information Overload & 4.5 & Informations clearly distributed and organized in a minimalist way almost in all pages.\\
     
    \end{tabular}
  \end{center}
\end{table}
\subsection{Information Overload}
Each information, both graphical and textual, is well balanced in each section of the website. This is one of the strength of website, even if in some sections the minimalism appears too grave and risks leading to a lack of information.

\section{Layout}
\label{Layout}

\begin{table}[H]
  \begin{center}
    \label{tab:table1}
    \begin{tabular}{||l|c|p{8cm}||} % <-- Alignments: 1st column left, 2nd middle and 3rd right, with vertical lines in between
      \textbf{Heuristic} & \textbf{Score} & \textbf{Comment}\\
      
      \hline
      Text Layout & 4 & The text is readable text and the font is appropriate in the various sections. Colour choice could be more improved.\\
      \hline
      Interaction Placeholder-Semiotics & 4 & Textual and visual labels are expressive and reflect a good interaction/effect aspect, excluded some cases in which there is an inappropriate response to that expected.\\
      \hline
      Interaction Placeholder-Consistency & 4 & Labels for interactive elements are well orgenized in term of position but the icon are not always consistent and can be slightly improved.\\
      \hline
      Spatial Allocation & 4 & Generally fine and appropriate allocation of contents in the various pages, amendable the organization of semantically-distant elements in some pages.\\
      \hline
      Consistency of Page Structure & 3 & Page structure of each topic is consistent among pages but refering to different groups we detected severe changes of structure.\\

    \end{tabular}
  \end{center}
\end{table}

\subsection{Text Layout}
Textual contents are easy-to-read. This is thanks to the font used in each point which is always proportional to the importance of the information. For instance, title or quotes has a larger font then simple description. About the choice for the colour of the font we note that, even if almost always it matches with the rest of the layout, in certain parts the choice of the colour in contrast with the background is not absolutely appropriate.
\subsection{Interaction Placeholder-Semiotics}

Textual and visual labels are expressive almost in every case. Only few situations link are not visible by underling or a hand-cursor. In addition we found relevant issues characterized by a strange behaviour in some case, for instance clicking on partners image in the Business Line section, as a result of which the page come back to the top of the page; moreover, this happens cliking on image TODO. After all, labels of interactive elements are consistent and reflect acceptably good the interaction and the effect.

Textual and visual labels are expressive almost in every case. Only few situations link are not visible by underling or a hand-cursor. In addition we found relevant issues caracterized by a strange behaviour in some case. For instance, clicking on partners image in the \textit{Business Line} section, the page come back at the top of the page. 
After all, labels of interactive elements are consistent and reflect acceptably good the interaction and the effect.


\begin{comment} 
Moreover, it happens that cliking on image TODO. 
\end{comment}

\subsection{Interaction Placeholder-Consistency}
Considering the website in its integrity, it doesn’t show so many problems in terms of wording, icon and position, so, given its easy of use, we have not find relevant violaitons. 
Anyway we highlight 2 issues:
\begin{itemize}
\item The \textbf{Business Lines} doesn’t always work as homepage;
\item Even if the labels are well organized in term of position, the choice of the icon are not always consistent.
\end{itemize}
\subsection{Spatial Allocation}
Each type of content is allocated spatially, in an appropriate way respect to the relevance. Semantically related elements are close to each other but we would appreciate a better organization in the screen and spacing of semantically distant elements; in particular we refer, for example, to the “Industries” section: different items should be space out a little. 
\subsection{Consistency of Page Structure}
Page structure of the topics is consistent among the pages, with few differences about contents allocation; but refering to different groups the structure often changes in a substantial way.
Anyway, in the resource section users can be redirected to other websites which have a completely different structure and font too; so they are absolutely not consistent, contrary to the topics of same sections.