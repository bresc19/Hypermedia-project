Regarding the first part of the \textit{Inspection}, as this usability evaluation study has highlighted, the Moviri’s website gives to users a quite good experience that could be easily improved by solving the found small issues.\\ 
For what concern Navigation, the strength is given by the Structural and Group Navigation:
\begin{itemize}
\item ensure a consistent aspect between pages; 
\item give an intuitive way to navigate between contents; 
\item overcome all navigation problems related to links’ miss between contents of a same; topic and structural problems. This aspect could be better managed by adding the possibility to go back to previous page (for example by including the followed path to reach each page); by adding links and letting users to move between similar contents without always accessing the header and by being more consistent in buttons layout and visibility (for example by always showing only the one related to the selected season).
\end{itemize}
Content have taken the best score between the 4 groups because the website contains lots of information well organized and without visible issues. The only adjustments that could be done is that sometimes the layout was found a little too rough but with a small effort could make pages look tidier and more clear.\\
The real Presentation weak point is given by the lack of consistency between the pages of different groups, first, and of the placeholders, not always consistent and that can be easily improved.\\
Regarding Nielsen’s Heuristics the strength point is the Efficiency of use, in fact we would like to say that the site is well done and ensures the user a good quality of the visit; the weak point, instead, is the visibility of the system status that is really lacking and demonstrates how much the developers have thought of a specialized user profile that has no problem finding himself in the various sections he wants to visit.
\par
Regarding the second part of the study, that concerns the \textit{User Testing}, we have already discussed into the results section the issued encountered by our testers but as conclusion we can say that our testers appreciated the website in its structure, layout and graphical elements and that they didn’t face big issues.\\
An evidence that we found both in the testers coming from Profile 1 and 2 regards the difficulty in identifying the landmark for the \textit{“Home”} section; as already underlined in the specific section of the system, this role is covered by \textit{"Business Lines"} button but without this being specified in any way, which leaves the user perplexed at first.\\
At the end, a suggestion that we can formulate by the analysis of all the data obtained by the study is that how as we had already seen, the web site can be easily consulted by a skilled user in the IT sector and becomes less and less easily visited by less experienced users for the complexity of the language that is used; we believe that these characteristics are predictable in a site that presents sectorial services and topics, designed for a well-defined user profile but some precautions already mentioned previously, such as the use of labels that clarify and anticipate the content of each section, could certainly make it usable to a much wider range of users.\\
We encounter some difficulties in aligning our ways of reasoning over the analysed aspects and also in getting the heuristics limits.\par 
We found this project kind of complex but in the same time interesting because it allowed us to understand in a clear way what has an impact over our sentiment as users and how we would like a website to be done.
