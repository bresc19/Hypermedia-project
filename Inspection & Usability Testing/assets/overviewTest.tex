In this part of the document we evaluate usability of Moviri basing on \textbf{User Testing}. User testing is a technique based on user-experience used to evaluate a product by testing it on selected users. Instead of inspection, this is more focused on human–computer interaction in order to eveluate the easyness of a website.

\section{Design of the study}
\subsection{User Profile}
User profile definition is very relevant for user testing. In fact, this help us to understand who can be recruited for user testing. The user profile detected is the following:
\begin{itemize}
\item \textit{Age}: between 25 and 45 years old;
\item \textit{Civil state}: irrelevant;
\item \textit{Technology capabilities}: General knowledge about Computer Science;
\end{itemize}

\subsection{Scenarios}
In this subsection are shown the goals of the user testing followed by the tasks needed to reach them:\par
\textbf{Scenario 1}: You need a job and you know that Moviri has a lots of open positions. You’re applying for a job opportunity as a Software.
\begin{itemize}
\item Visit the website section for career service;
\item Start the application procedure;
\item Choose a job offer that is suitable for your background (such as Software Engineering); 
\item Select the position for which you want to run for;
\item Insert your data in the form for the application;
\end{itemize}\par
\textbf{Scenario 2}: you’re looking for a white paper related to a particular service/solution for your company.
\begin{itemize}
\item Visit the website section for consulting Moviri’s resource;
\item Find and select the white paper that you need;
\item Fill out the form with your information in order to receive the white paper by e-mail;
\end{itemize}\par
\textbf{Scenario 3}: you’re looking for a initiative called “Keep IT Up” and you want to keep current on it.
\begin{itemize}
\item Visit the website section relevant to news;
\item Find the article about “Keep IT Up” initiative and select it;
\item After you get some info from the article, in order to stay up fill out the form;
\end{itemize}\par

\subsection{Variable Measure}
In order to evaluate user testing on some metrics, usability variables are defined:
\begin{itemize}
\item \textbf{Time of execution}: it's measure in \textit{seconds} and represents the time spent on a given task. It starts from the moment in which user begin focusing his attention on it;
\item \textbf{Effectiveness}: it represents the task success rate. It's measured by a value \textbf{between 0 and 1}:
    \begin{itemize} 
    \item \textit{1.0}: task completed with success;
    \item \textit{0.5}: task partially completed;
    \item \textit{0.0}: task is not completed;
    \end{itemize}
\item \textbf{Errors}: it's measured by an integer representing the number of errors made during the execution on a given task;
\item \textbf{Perceived tasks difficulty}: it's expressed by an integer \textit{between 0 and 5} given by an user after the completion of a given task;
\item \textbf{Satisfaction}: it's given by an integer \textit{between 0 and 5} provided by user immediately after the task execution;
\end{itemize}


\subsection{Final Survey}
After the execution of every tasks, a final survey is provided to users in order to have additional collected data. This provides extra feedbacks about aspects already highlighted during the inspection.
The questions, provided through a survey made with Microsoft Forms, are the following:
