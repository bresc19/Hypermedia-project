In this part of the document we're focusing on the evaluation of usability of \textit{Moviri} through \textbf{inspection}. Inspection allow us to find usability issues and obstacles for the user when interacting with a web application. In particular, this is done thanks to \textbf{heuristics} which guide the expert to explore the website and check compliance with usability principles.
\section{Goals}
Before inspection, goals are defined in order to deeply inspect the website and to focus on the main aspect. 
\begin{itemize}
\item Read experiences of other companies and changes adopted;
\item Find the appropirate technology needed;
\item Interact with Moviri due to become a new customer;
\end{itemize}
\section{Inspection method}

We decide to adopt \textbf{MiLE heuristics} in order to inspect the website. These are divided into different categories relevant to a particular aspect.

\textbf{Navigation}: It aims to evaluate the easiness with which an user navigates into each part of the website.
\begin{itemize} 
\item \textbf{Interaction consistency}: do pages of the same type have the same links and interaction capability?
\item \textbf{Group navigation}: is it easy to navigate from and among groups of
“items”?
\item \textbf{Structural Navigation}: is it easy to navigate among the semantic components of a Topic?
\item \textbf{Semantic Navigation}: is it easy to navigate among group members and from a group introductory page to group members (and the other way around)?
\item \textbf{Landmarks}: is it easy to navigate from a Topic to a related one?
\end{itemize}

\textbf{Contents}: It indicates how in the website information is well balanced in each page and section.  
\begin{itemize}
\item \textbf{Information Overload}: is the information in a page too much or too little and does it fit the page layout?
\end{itemize}

\textbf{Layout}: It serves to estimate if the website is graphically expressive enough and readable.TODO
\begin{itemize}
\item \textbf{Text Layout}: is the text readable? Is font size appropriate?
\item \textbf{Interaction Placeholder-Semiotics}: are textual or visual labels of interactive elements “expressive”? i.e., do they reflect the meaning of the interaction and its effects? Are they consistent?
\item \textbf{Interaction Placeholders-Consistency}: are textual or visual labels of interactive elements consistent in terms of wording, icon, position, etc.?
\item \textbf{Spatial Allocation}: is the on-screen allocation of contents and visual appropriate for their relevance? Are “semantically related” elements close and “semantically distant” element far away?
\item \textbf{Consistency of Page Structure}: do pages of the same type have the same lay out (same visual properties of each component and similar lay-out organization of the various elements?)
\end{itemize} 


\section{Scoring metric}
Before the inspection, a metric is defined in order to evaluate each heuristic. The evaluation consist in the assigment of a score from 0 to 5. The following image gives an explanation of each score.

\begin{itemize}
\item \textbf{0}: Many severe violations are detected;
\item \textbf{1}: Some severe violations are detected;
\item \textbf{2}: Few violations are detected;
\item \textbf{3}: Small issues are detected;
\item \textbf{4}: No issue are detected. The heuristic is satisfied;
\item \textbf{5}: No issue are detected. The heuristic is fully satisfied; 
\end{itemize}


foto + click che porta in alto